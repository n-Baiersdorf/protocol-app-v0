\documentclass{article}%
\usepackage[T1]{fontenc}%
\usepackage[utf8]{inputenc}%
\usepackage{lmodern}%
\usepackage{textcomp}%
\usepackage{lastpage}%
\usepackage{geometry}%
\geometry{head=40pt,margin=2.5cm,bottom=2.5cm,includeheadfoot=True}%
\usepackage[ngerman]{babel}%
\usepackage{amsmath}%
\usepackage{amssymb}%
\usepackage{graphicx}%
\usepackage{float}%
\usepackage{fancyhdr}%
\usepackage{hyperref}%
%
%
%
\begin{document}%
\normalsize%
\pagestyle{fancy}%
\fancyhf{}%
\fancyhead{[L]{Laborprotokoll}}%
\fancyhead{[R]{\today}}%
\fancyfoot{[C]{\thepage\ von \pageref{LastPage}}}%
\section{Zielsetzung}%
\label{sec:Zielsetzung}%
* Herstellung eines Löffels aus Säurecasein

%
\section{Theoretischer Hintergrund}%
\label{sec:TheoretischerHintergrund}%
* Säurecasein ist ein Gemisch aus Essigsäure und Kalilauge, das in der Regel für die Herstellung von Löffeln verwendet wird.
* Phenolphthalein ist ein Indikator, der den Umschlag von sauren bis basischen Lösungen verfolgt.

%
\section{Materialien und Geräte}%
\label{sec:MaterialienundGerte}%
* Säurecasein (0.1 M)
* HCl-Lösung (unbekannte Konzentration)
* Phenolphthalein
* Mikroskop

%
\section{Durchführung}%
\label{sec:Durchfhrung}%
1. In einem Becher geben wir 23,5 mL NaOH-Lösung hinzu und verraten die Lösung auf eine Temperatur von 20 °C.
2. In einem anderen Becher geben wir die unbekannte HCl-Lösung hinzu und verraten die Lösung auf die gleichen Bedingungen wie oben.
3. Wir mischen die beiden Lösungen zusammen und beobachten den Umschlag des Indikators Phenolphthalein.
4. Wir berühren den Indikator mit einem Mikroskop, um den Umschlag zu beobachten.

%
\section{Ergebnisse und Beobachtungen}%
\label{sec:ErgebnisseundBeobachtungen}%
* Der Indikator Phenolphthalein ändert seine Farbe von farblos zu rosa bei der Reaktion.
* Die Lösung bleibt stabil rosa und es gibt keine Veränderungen im Umschlagsprozess.

%
\section{Diskussion}%
\label{sec:Diskussion}%
* Die Umschläge des Indikators Phenolphthalein belegen, dass die Reaktion zwischen Säurecasein und HCl erfolgreich ist.
* Die Stabilität der rosa Farbe des Indikators belegt, dass die Lösung stabil bleibt.

%
\section{Schlussfolgerung}%
\label{sec:Schlussfolgerung}%
* Die Herstellung von Löffeln aus Säurecasein ist möglich und erbringt den gewünschten Umschlag des Indikators Phenolphthalein.
* Die Verwendung von HCl kann die Stabilität der Lösung verbessern, indem sie das Gleichgewicht der Reaktion beeinflusst.

%
\end{document}