\documentclass{article}%
\usepackage[T1]{fontenc}%
\usepackage[utf8]{inputenc}%
\usepackage{lmodern}%
\usepackage{textcomp}%
\usepackage{lastpage}%
\usepackage{geometry}%
\geometry{head=40pt,margin=2.5cm,bottom=2.5cm,includeheadfoot=True}%
\usepackage[ngerman]{babel}%
\usepackage{amsmath}%
\usepackage{amssymb}%
\usepackage{graphicx}%
\usepackage{float}%
\usepackage{fancyhdr}%
\usepackage{hyperref}%
%
%
%
\begin{document}%
\normalsize%
\pagestyle{fancy}%
\fancyhf{}%
\fancyhead{[L]{Laborprotokoll}}%
\fancyhead{[R]{\today}}%
\fancyfoot{[C]{\thepage\ von \pageref{LastPage}}}%
\section{Theoretischer Hintergrund}%
\label{sec:TheoretischerHintergrund}%
Casein ist ein Protein, das in Milch enthalten ist und wichtige Eigenschaften wie Festigkeit und Stabilität aufweist. Die Extraktion von Casein kann für verschiedene Anwendungen verwendet werden, wie z.B. die Herstellung von Käse oder Milchpulver.

%
\section{Materialien und Geräte}%
\label{sec:MaterialienundGerte}%
* Milch (Untergruppe: Cow-Milch)
* NaOH-Lösung (0.1 M)
* HCl-Läsion (unbekannte Konzentration)
* Phenolphthalein-Indikator
* Gefäß für die Lösung (25 mL)
* Röhrenkanne (250 mL)
* Schraubensorter

%
\section{Durchführung}%
\label{sec:Durchfhrung}%
1. Geben 23,5 mL NaOH-Lösung in das Gefäß und mixen es mit einem Schraubensorter.
2. Geben die unbekannte HCl-Lösung in das Röhrenkanne und mixen sie mit dem NaOH-Gefäß.
3. Lassen die Mischung für 10 Minuten reagieren, beobachtende die Entwicklung des Indikators.
4. Nehmen die Mischung aus dem Röhrenkanne und füllen das Gefäß mit der Lösung.
5. Beobachten den Umschlag des Indikators und notieren sich die Farbe.

%
\section{Ergebnisse und Beobachtungen}%
\label{sec:ErgebnisseundBeobachtungen}%
Die Mischung entwickelt eine farbige Lösung, die rosa gefärbt ist. Der Indikator Phenolphthalein ändert seine Farbe von farblos zu rosa bei der Reaktion.

%
\section{Diskussion}%
\label{sec:Diskussion}%
Die Extraktion von Casein kann durch die Verwendung von NaOH und HCl erfolgen. Die Konzentration des HCl-Gefäßes ist unbekannt, daher ist es schwierig zu sagen, ob die Extraktion erfolgreich war. Eine mögliche Möglichkeit besteht darin, die HCl-Konzentration durch Titration zu bestimmen und anschließend die Extraktion zu durchführen.

%
\end{document}